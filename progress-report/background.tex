\section{Research}
\label{sec:background}
%literature review
\subsection{Learning Chisel}
As the HDL that is used in RocketChip\cite{}, the underlying SoC generator that this project utilises, being able to write and understand Chisel to at least a basic extent is necessary. The Chisel Learning Journey\cite{} is a step-by-step guide to quickly becoming effective at writing Chisel and producing effective code. This has been incredibly useful in understanding the vivado-risc-v project and RocketChip generator, though has required a large amount of time. I have not yet fully completed the guide, but have achieved a basic proficiency with Scala that has made creating new SoC designs possible, even if no complex topics have been covered.

\subsection{Existing Project}
The documentation for the existing project is minimal at best. There is adequate to generate existing designs and implement on an FPGA after a few hours of reading and compilation, but little exists after this. This is somewhat expected - the project is open-source, but primarily for usage by the maintainers. A lot of the time spent has been in understanding the existing project, as well as creating a development environment that matches what is required. 

\subsubsection{RocketChip}
Much time has also been spent in reading RocketChip documentation\cite{}. As the SoC generator used by this project, understanding it's capabilities is key to effectively using it. The documentation is very sparse and mostly points to that of the sub-repositories which are used extensively in the generator. Code analysis of the RocketChip generator has been the most effective way to learn about the generator, but this has not been entirely successful. The code is often uncommented and assumes a great deal of knowledge about CPU design, as well as the RISC-V ISA, making reading it very difficult. Due to the niche 

\subsection{RISC-V ISA and Assembly}
There is excellent documentation on the RISC-V ISA, hosted by the RISC-V foundation. There are 2 documents, the unprivileged specification and privileged specification. The unprivileged specification details instructions that can be executed in U-mode, or user mode - when user programs are executing. The privileged specification details instructions that are executed in M-mode, or machine mode and are intended to be called by the OS or bootrom/bootloader. The documents describe all instructions and registers on RISC-V cores, but do not contain examples or context for many of them. In addition to this, there is huge variance between cores - parts of the RISC-V ISA are optional and are implementation specific, describing what behaviour should be seen as opposed to how much of it is done.

There are also few sources for writing RISC-V Assembly code. TODO: write more about RISC-V assembly and why im going to be using it.