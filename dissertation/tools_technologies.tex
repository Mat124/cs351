\chapter{Tools and Technologies}
\label{ch:tools_technologies}
\section{FPGA Board}
The 

\section{Chisel & Vivado}
\section{vivado-risc-v and RocketChip} % should maybe be in background?
To design the SoC, a generator will be used. The aim of the project is to design 

This projects extends an existing repository vivado-risc-v\cite{vivado-risc-v}. The vivado-risc-v repository is a collection of other tools and resources, along with additional scripts and sources, that can generate HDL (hardware description language) for RISC-V based SoCs, link them to hardware resources on specific FPGAs using Vivado and create the bitstream to program the FPGA. The repository also has scripts to generate a bootable external memory device, such as an SD card, with OpenSBI, U-Boot, Linux kernel and Debian OS. This allows a functional RISC-V SoC to be instantiated on an FPGA, boot into Debian Linux and be used as a general-purpose computer, albeit without GUI support.

The main use of vivado-risc-v is board support and Vivado project generation. The HDL for the SoC is generated by other tools, such as RocketChip\cite{rocketchip} or BOOM\cite{boom-core}, but the files then need to be added to a created Vivado project

\section{Bare-metal code}
\section{Linux}
\section{Git}