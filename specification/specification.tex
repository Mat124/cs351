\documentclass[a4paper,fleqn,11pt]{article}

%%%%%%%%%%%%%%%%%%%%

\input{../common/common.tex}
%%%%%%%%%%%%%%%%%%%%%%%%%%%%%%%%%%%%%%%%%%%%%%%%%%%%%%%%%%%%%%%%%%%%%%%%%%%%%%%
%% Project-specific configuration
%%%%%%%%%%%%%%%%%%%%%%%%%%%%%%%%%%%%%%%%%%%%%%%%%%%%%%%%%%%%%%%%%%%%%%%%%%%%%%%

\author{Matthew Knight}
\title{Creating a RISC-V heterogeneous CPU architecture}
% \supervisor{Your supervisor's name}
% \yearofstudy{3\textsuperscript{rd}}

%%%%%%%%%%%%%%%%%%%%%%%%%%%%%%%%%%%%%%%%%%%%%%%%%%%%%%%%%%%%%%%%%%%%%%%%%%%%%%%


\assignment{Project specification}

%%%%%%%%%%%%%%%%%%%%

\pagestyle{plain}
\renewcommand{\headrulewidth}{0.0pt}

\makeatletter
\fancypagestyle{plain}{
	\fancyhf{}
	\fancyhead[R]{\textit{\@title} - \textit{\@assignment}}
    \fancyhead[L]{\textit{\@author}}
    \fancyfoot[C]{\thepage}
}
\makeatother

%%%%%%%%%%%%%%%%%%%%

\begin{document}

\input{../common/titlepage.tex}

\pagestyle{plain}

\section{Introduction}
The aim of this project is to design a heterogeneous CPU using the RISC-V open source instruction set. Once designed, the CPU should be implementable on an FPGA and be able to run a Debian Linux distro. 

\subsection{Definitions}
\begin{description}
    \item[Heterogeneous CPU] A CPU that contains 2 different types of core, mismatched in performance and power consumption.
    \item[RISC-V ISA] An open-source CPU instruction set architecture.
    \item[SoC] System-on-Chip, a complete computing system within a single circuit board or design. 
    \item[FPGA] A field programmable gate array, an integrated circuit that can be reprogrammed using a hardware description language (HDL).
\end{description}

\section{Background}
Designing a CPU is a complex challenge. They need to be power efficient to reduce the cost of running the system, as well as providing adequate processing power when required. Often one is sacrificed to improve the other: for instance, energy efficiency is often reduced due in larger CPU designs. This is because more power has to be supplied to achieve the same performance that a smaller processor could achieve while using less power, due to the reduced size.

A heterogeneous CPU attempts to solve this issue by combining dissimilar core designs; often a powerful core, or p-core, and an efficient core, or e-core. When only light processing is required, the p-core can be effectively shutdown and the e-core will do all processing, resulting in less power used. For heavy processing, the p-core is then used to increase peak performance. Depending on the exact implementation, the p-core can be used either individually or in tandem with the e-core, but both result in greater performance than just the e-core.

This has been used extensively in many mobile devices. ARM released the big.LITTLE in 2011, the first mobile heterogeneous architecture. The architecture provides the low power usage needed in mobile devices the majority of the time when idle, only running tasks like checking for new messages. It also provides the high performance that can be demanded from phones when used to browse the internet, play mobile games, etc. Most big.LITTLE designs use HMP (heterogeneous multiprocessing) where all cores are available to have processes assigned to them at all times. The alternatives to this are: clustered switching, where either the big cores or the LITTLE cores are in use, and in-kernel switching, where big and LITTLE cores are paired to form a virtual core and only one performs the tasks assigned to the virtual core at any one time.

\section{Existing Solutions}
https://ieeexplore.ieee.org/document/8702538 - Low Energy Heterogeneous Computing with Multiple RISC-V and CGRA Cores. This implements only a single type of RISC-V core which then uses a CGRA (Coarse-grained reconfigurable array) as an accelerator in certain tasks. This is dissimilar to what the project is proposing.

https://www.mdpi.com/1424-8220/21/19/6491 - A Heterogeneous RISC-V Processor for Efficient DNN Application in Smart Sensing System. This is similar to the above, where a RISC-V core is used as the host to an accelerator designed for specific workloads. This is also dissimilar to what the project is proposing.

https://arxiv.org/abs/2206.01901 - Enabling Heterogeneous, Multicore SoC Research with RISC-V and ESP. An extension to the ESP (Embedded Scalable Platform) project to enable the creation of an SoC with multiple RISC-V cores. This does not allow the creation of an SoC with different types of RISC-V core, but does implement some of the functions required for this and so could potentially be extended to allow for dissimilar RISC-V cores implemented in a single SoC.

https://open-src-soc.org/2022-05/media/posters/4th-RISC-V-Meeting-2022-05-03-Luca-Valente-poster-abstract.pdf - 64-bit RISC-V host core running Linux that offloads tasks to a PMCA (Programmable ManyCore Accelerator) made of 8 32-bit RISC-V cores. This is the most similar work I could find to the aim of the project, but differs by having the smaller RISC-V cores as accelerators as opposed to linux-capable, indepedent cores.

Most existing heterogeneous solutions appear to comprise of a 'host' processor and an accelerator, where the host processor is constantly running and offloads tasks to the accelerator when needed. This is different to the aim of this project, where the cores are of equal standing and both could be used individually. Multicore RISC-V SoCs that implement SMP in Linux are most similar to my objectives - this is essentially what the aims are, with the extra of the cores being two different types and performance levels.

\section{Objectives}
The overall objective is design a heterogeneous RISC-V CPU to be used in an SoC that can run Linux and use both cores to improve peak performance and power efficiency. While this is the overarching goal, it can be broken down into smaller objectives.
\begin{enumerate}
    \item Design a heterogeneous RISC-V SoC containing 2 dissimilar cores, capable of executing at least the RV64I instruction set.
    \item Run Linux on the previously designed SoC and connect to it via SSH.
    \item Allow processes to execute on both cores inside of Linux.
    \item Intelligently select which core the process will run on, depending on factors such as process priority and resource usage.
    \item Allow both cores to simultaneously execute processes when required.
    \item Have comparable or better performance per watt than an SoC with only 1 of the larger cores.
\end{enumerate}

\section{Methods and Resources}
https://esp.cs.columbia.edu/ - open source SoC platform, can be used to develop heterogeneous CPU architectures (not RISC-V specific). This allows accelerators and host cores to be implemented together, but does not allow the creation of an SoC with more than 1 type of core, or even multi-core RISC-V SoCs without the extension listed previously. I believe this platform is therefore unsuitable for use in my project due to the large number of changes required and the limited time available for my project.

https://github.com/ucb-bar/chipyard - Chipyard is an open source framework for the development of chisel-based SoC's. It allows the creation of multi-core RISC-V SoCs with various core types, including allowing multiple types of core to be implemented together on the same SoC. Chipyard designs that meet the minimum specifications for the Linux kernel are also able to run Linux once loaded onto an FPGA, but only certain types of board are supported. It is also unclear whether the heterogeneous designs made using Chipyard would be able to run Linux, as I have been unable to find any examples of heterogeneous RISC-V CPU running Linux.

https://github.com/eugene-tarassov/vivado-risc-v - Scripts and sources to design a various RISC-V SoCs that can run Linux. Allows for various designs of multicore RISC-V systems, featuring Rocket cores (https://www2.eecs.berkeley.edu/Pubs/TechRpts/2016/EECS-2016-17.html) and BOOM cores (https://github.com/riscv-boom/riscv-boom). The base configuration options given do not support heterogeneous RISC-V SoC, but the project is based on the Rocket Chip generator. This directly supports heterogenous processing at various levels. The Rocket Custom Coprocessor (RoCC) interface is used primarily for application specific coprocessors. Tile and TileLink are 
generators multiple processor agents, with options for coherence, shared memory and shared networks.

The development of the Linux-capable heterogeneous RISC-V CPU will most likely be done by modifying

\section{Risks and Ethical Considerations}
\begin{center}
    \begin{tabular}{|c|c|c|c|}
        \hline
        Risk & Risk Description & Mitigation & Risk level \\
        \hline
        FPGA is too small & As the design for the SoC grows larger and more complex, it will require a greater amount of LUTs. This may result in a larger design than the FPGA to be used can implement. & This risk could be mitigated by purchasing a larger FPGA, or by reducing the complexity of the design and objectives. & High \\
        \hline
        Student is unable to work & Either by illness or other matters, the student is unable to continue work on the project for an extended period. & Mitigating this risk is difficult as illness cannot be predicted, but the likelihood of this occurring is very low. & Low \\
        \hline
        Concept is flawed/impossible/too difficult & As the project progresses, it appears that the main objectives of the project cannot be completed either at all, or in the time-frame provided. & The objectives and scope of the project can be changed/reduced in order to produce a full piece of work by the deadline. & Low \\
    \end{tabular}
\end{center}

There are no ethical considerations for this project.

\section{Timetable}
I'll make a gantt chart and put it here.

% \chapter{Introduction}
\label{ch:introduction}

The aim of this project is to design a heterogeneous system on a chip (SoC) using the RISC-V open standard ISA\cite{riscv-1}\cite{riscv-2}. Once designed, the SoC should be implementable on an FPGA and be able to execute bare-metal C code and assembly with simultaneous multi-processing (SMP), allowing all cores to execute code at the same time.

\section{Motivation}
Designing a processor is a complex challenge. They need to be power efficient to reduce the cost of running the system, as well as providing adequate processing power when required. Often one is sacrificed to improve the other: for instance, energy efficiency is often reduced due in designs of large processors containing multiple cores. Due to the increased size and complexity, more power has to be supplied to achieve the same performance as a smaller processor.

A heterogeneous CPU attempts to solve this issue by combining dissimilar core designs; often a big, powerful core, or B core, and a small, efficient core, or S core. When only light processing is required, the B core can be effectively shutdown and the S core will do all processing, resulting in less power used. For heavy processing, the B core is then used to increase peak performance. Depending on the exact implementation, the B core can be used either individually or in tandem with the S core, but both result in greater performance than just the S core, or greater than two S cores if used together.

This has been used extensively in many mobile devices, and increasingly in larger devices like laptops. The Apple M2 processor\cite{applem2} is one such example, implementing 4 high-performance cores and 4 energy-efficient cores. These are arranged in a hybrid configuration similar to ARM DynamIQ and big.LITTLE\cite{biglittle}, allowing cores to dynamically be assigned work that best fits them. In Apple's design, both cores are capable of executing the same code, with the performance core benefitting from much increased cache and other proprietary changes to increase it's speed over the efficiency core. 

\clearpage

\section{Objectives}
%One sentence summary of your project. Followed by a short list of concrete objectives:
The overall aim is to design a heterogeneous SoC containing two types of RISC-V core that can execute bare-metal C code and assembly with SMP. Objectives have been labelled according to the MoSCoW method\cite{case-method-fasttrack} of must, could, should and won't to indicate their importance and expected completion.
\begin{figure}[h!]
    \centering
    \begin{enumerate}
        \item Design a heterogeneous RISC-V SoC containing 2 dissimilar cores, each capable of executing at least the RV64 instruction set (Must).
        \item Execute assembly instructions on both cores when implemented in an SoC on an FPGA (Must).
        \item Execute bare-metal code on both cores when implemented in an SoC on an FPGA (Should).
        \item Measure the performance of the SoC for comparison against homogeneous designs. (Should)
        \item Run embedded Linux on the SoC and connect to it via serial or SSH (Could).
        \item Allow processes to execute on both cores inside of embedded Linux (Could).
        \item Intelligently select which core the process will run on, depending on factors such as process priority and resource usage (Won't).
    \end{enumerate}
    \caption{Objectives for the project}
    \label{fig:objectives}
\end{figure}
% \chapter{Background and Research}
\label{ch:background}

\section{RISC-V}
\label{sec:riscv-background}
\subsection{ISAs and RISC vs CISC}
An ISA is an instruction set architecture, and is the formal definition for what an abstract model of a CPU. The ISA defines the instructions, registers, communication standards and other features of a CPU in order to allow a person with the ISA to design a physical CPU that would implement the abstract model. CPUs that implement the same ISA (or supersets of an ISA) are able to execute the same code, meaning programs written for a CPU can be run on a different CPU if they share the same ISA. This is how a single compiled program is able to be run on so many different systems, and is incredibly important for modern devices, where there is a huge range of processors available. 

Many modern ISAs are based on older ISAs that have been extended to allow for new instructions, registers, communication protocols, etc, while maintaining support for previous ISAs. This is typically called CISC (Complex Instruction Set Computers), where there are a very large amount of instructions the processor can execute. The backwards compatibility can be very useful, and having explicit instructions for many tasks can mean the tasks are completed more efficiently than if multiple smaller instructions were executed. Having large instructions can also reduce program sizes, due to the increased operations per instruction, which is very beneficial, especially in small systems.

There are several issues with CISC however, such as increased complexity in the CPU as the amount of instructions increases, leading to rising costs of design and manufacture, and can force the physical size of the CPU to be larger to accommodate the extra logic space for larger instructions. More complex designs will also make future development harder if backwards compatibility is to be maintained. 

The alternative to CISC is RISC (Reduced Instruction Set Computer). RISC aims to reduce the number of instructions the CPU requires in order to keep the design as simple as possible. This allows the design to be done much more easily, as well as allowing future extensions to be easier. Reduced complexity can also allow for greater optimisations in what instructions are in the ISA, with the aim of achieving greater performance than CISC by executing more instructions at much greater speed. There are drawbacks however, with increased program size and possible reduction in speed of execution for multiple operations that could've been completed in a single instruction using CISC.

\subsection{The RISC-V ISA}
RISC-V is a new, modern, open-standard ISA that follows RISC concepts. The ISA does not specify a single instruction set, but multiple by having three different width instruction sets with different base instructions. 32-bit RISC-V (RV32I) is focussed on embedded and personal systems, 64-bit RISC-V (RV64I) for personal and server systems and 128-bit RISC-V (RV128I) for server and high-performance compute systems. RV32 and RV64 have embedded versions, RV32E and RV64E, that have reduced registers and instructions.

\subsubsection{Extensions}
At base, these implement only integer addition/subtraction. The ISA contains extensions that add more functionality to the CPUs, such as support for multiplication/division, floating point operations IEEE-754, compressed instructions, atomics, etc. An instruction set implementing these extensions is referred to by adding the initial of the extension to the name, so a 64-bit CPU with integer addition/subtraction, multiplication/division and compressed instructions would be `RV64IMC`. Custom extensions can also be written, allowing for a great deal of customisation in RISC-V CPU designs.

There are 30 official extensions to the base instruction sets. The most popular are listed below.
\begin{description}
    \item[M] Integer multiplication and division
    \item[A] Atomic instructions
    \item[F] Single-precision floating point
    \item[D] Double-precision floating point
    \item[Zicsr] Control and Status Register (CSR)
    \item[Zifencei] Instruction-Fetch Fence
    \item[G] Short for \texttt{IMAFDZicsrZifencei} as a 'general-purpose' CPU, e.g. RV64G
    \item[C] Compressed instructions, 16-bit instead of 32/64
    \item[Others] Quad-precision floating point, bit manipulation, vector, misaligned atomics, etc
\end{description}

\section{Processor Design}
Processor performance can be measured in multiple ways, with the main metrics being speed at executing tasks, cost of manufacturing, physical size and energy consumption. These are implicitly linked - a processor with larger physical size will likely draw a larger amount of energy, contain more logic and so be faster at executing tasks and cost more to manufacture. The aim of processor design is to maximise the speed of task execution, while minimising the rest. The balance between these is what gives rise to different processor designs, as a mobile device is much more constrained in power usage than a large server and will require a different processor.

\subsection{Power reduction in processors}
Maximising task execution while minimising energy consumption can be balanced in a multitude of ways. Power consumption in CPUs can be split into two sections: switching power and leakage power. Switching power is the power used by CMOS (transistor) gates as they change states, and varies on CPU activity. Leakage power is the power lost through slow leakage current flow through transistors in the 'off' state, and has increased as transistor sizes decrease and the boundary that must be crossed reduces.

\subsubsection*{Dynamic Voltage Frequency Scaling (DVFS)}
CPUs contain clocks, which synchronises the components inside the CPU as they change state and execute instructions. Increasing the clock speed will result in an increase to the number of instructions executed per second, directly increasing the performance of the CPU. However, this directly increases the amount of switching power the CPU draws, given by equation \ref*{eq:pswitching}.

\begin{equation} \label{eq:pswitching}
    P_{switching} = \alpha \cdot C \cdot V^2 \cdot f
    \alpha = activity factor, proportion of transistors that switch every cycle
    C = capacitance switched per cycle
    V = transistor supply voltage
    f = clock frequency
\end{equation}

In addition to this, increased clock speeds often require increased voltage in order to increase the current flow through transistors to activate them in the reduced time frame from shorter clock periods. Therefore, reducing the clock speed and the voltage will result in a quadratic reduction in switching power for a linear reduction in performance and if low performance is needed, the power usage can be significantly reduced. CPUs implementing dynamic voltage frequency scaling are able to adjust their voltage and frequency during runtime dependant on the current tasks being run. This allows them to dramatically reduce power consumption during periods of low CPU utilisation, while still being able to provide good performance with high CPU utilisation.

However, this makes some requirements of the CPU. There must be software that allows the CPU to estimate it's current utilisation, and this has to be run frequently in order to have a fast response and increase clock speed when a demanding task is run. This can take up valuable CPU time, and requires some form of scheduling to repeatedly switch to and from this task. The CPU must also implement hardware that allows it to change the clock speed and voltage, increasing the physical size of the CPU and increasing manufacturing costs.

\subsubsection{Clock Gating}
Clock gating is another technique for reducing power consumption in a CPU. When a section of the CPU is unused for a number of cycles, the clock signal to that section is removed. This disconnect reduces the switching power, as no transistor switching will occur in that section of the CPU without the clock, as well as reducing the capacitance seen by the clock generator.

\subsection{Heterogeneous Designs}
Another solution for power reduction in processors is heterogeneous designs. Multicore processors contain multiple CPUs in order to increase the potential performance in parallel computing tasks. A heterogeneous design contains multiple, different CPU designs instead of all CPUs being copies of each other as in homogeneous designs.

By having multiple different types of CPU, the processor can increase efficiency and reduce power consumption by intelligently selecting which CPU to run a task on. For example, a processor containing a small (S) core and a big (B) core could be running in a mobile phone. For the vast majority of the time, the mobile phone is not in use and only runs tasks such as checking for texts, emails, etc. These tasks can be run on the S core, with the B core fully disabled/clock gated, reducing the power consumption to that of the S core + B core leakage current. This power consumption should be less than that of the B core performing the tasks, in order for the processor to actually provide efficiency increases over a B core homogeneous design.

When the mobile device is actively in use, like video playback or mobile games, the B core would be used to provide greater performance than the S core, or both used for parallel tasks. This allows a heterogeneous S+B design to provide better energy efficiency and better performance than a homogeneous design with one B core.

There are some drawbacks to heterogeneous designs. In order to properly utilise the differences between the cores, a scheduler must be customised for the exact processor as other designs with different CPUs will need to switch which task is run where at different stages. A heterogeneous system with a medium core and a big core will be able to use the smaller core for more intensive tasks than a heterogeneous system with a small core and a big core, thus requiring a scheduler with different parameters.

The physical size and complexity of the processor will also be increased compared to a single core design. This increases the cost of design and manufacturing, another drawback compared to homogeneous designs.

\subsubsection{Accelerator-style Heterogeneous Designs}
Some heterogeneous designs do not attempt to pair types of general purpose CPUs, but instead have a host CPU type and accelerator CPU type. The host CPU explicitly schedules tasks for the accelerator CPU, which is typically optimised explicitly for a certain task to increase the performance and efficiency in completing it.

\section{Related work}
%TODO: To be updated - mostly copied from specification, want this to be much more critical/analytical
%Discuss related work.
\subsection{A RISC-V Heterogeneous SoC for Embedded Devices\cite{valenterisc}}
This project is ongoing, and presents work designing a RV64 (RISC-V 64-bit) host core that offloads tasks to a PMCA (Programmable Many Core Accelerator) made from RV32 (RISC-V 32-bit) cores, which implement extensions for machine learning and discrete signal processing. The suggested use-case for the SoC is in IoT applications and programmable embedded devices. The host core is Linux compatible, and offers a full OS that acts as a platform for programs that run on the PMCA. The use of a large RV64 core to allow a full Linux OS to run on the SoC provides a huge amount of flexibility to the programmer, as the OS implements features like CLI, memory virtualisation, networking and more that allow programs to be written much more generally than embedded software running without an OS. However, this usage of a full Linux OS could be considered excessive for the use-case. An embedded Linux OS would have a lower overhead due to the reduced services it offers, which is very beneficial in an embedded environment where efficiency is highly important. Unfortunately, no data is provided about the processing power or energy usage of the design.

\subsection{Muntjac multicore RV64 processor\cite{UCAM-CL-TR-972}}
Muntjac is an SoC generator comprising of multiple components, that can be used to produce a Linux capable SoC. There is only one type of core in the system, RV64, but the SoC can be multicore. The project report is dedicated to the design of the core and cache as opposed to usage in any devices, as the purpose is to provide an easily understood and extensible platform for specialised designs. This is excellently done - the project is very well presented and uses multiple open-standards like TileLink\cite{tilelink} to increase the ease of working with it. It would be possible for this project to be extended for a heterogeneous SoC design, but it appears that this has not yet been done in any public works that extend the project.
% \section{Progress}
\label{sec:progress}
%Summarise the progress you have made so far. You can cross-reference other sections (\Cref{sec:background}).
\subsection{Sourcing an FPGA}
It was quickly identified that a larger FPGA was required in order to achieve the full project aims. The current Nexys A7-100T\cite{nexys-a7-100t} has 15,850 slices, each of which contain 4 LUTs for a total of 63,400. This is inadequate, as the SoC requires 10,800 LUTs + 27,500 LUTs per "big" RV64 rocketcore, preventing more than 1 "big" RV64 rocketcore being implemented on the FPGA.

The other option for a larger core is the Sonic BOOM RV64 cores. These have a much larger LUT requirement, with the "medium" BOOM\cite{boom-core} RV64 core using 148,500 LUTs, which is clearly not possible to implement on the current FPGA.

Alternative boards are the Nexys Video and Genesys 2. The Nexys Video contains 134,600 LUTs and would allow for multiple "big" rocketcores to be implemented, but would be unlikely to fit any BOOM cores. The Genesys 2 contains 203,800 LUTs and would allow for a BOOM and "big" rocketcore to be implemented at once. Due to monetary reasons, the Nexys Video was chosen as the FPGA to be sourced. The board is currently out of stock from University sources, so no progress has been made towards purchasing it.

\subsection{SoC Design}
An initial SoC design has been created, consisting of a large and small core. This is using the rocketchip\cite{rocketchip} RV64 cores, and are generated using a Chisel description of the processor, with the large core being the "big" rocketcore and the small core the "small" rocketcore. The difference between these are changes to the instruction set extensions implemented and the size of the instruction and data caches. This change reduces the core footprint and LUT utilisation on the FPGA significantly, from an estimated 27500 LUTs to only 7600 LUTs. This is a significant reduction, though comes with the cost of reduced functionality in the smaller core.

\begin{figure}[h!]
    \centering
    \includegraphics[scale=0.5]{./img/SoC_b1.png}
    \caption{Hardware utilisation of SoC with 1 "big" core}
    \label{fig:b1-util}
\end{figure}

\begin{figure}[h!]
    \centering
    \includegraphics[scale=0.5]{./img/SoC_b1s1.png}
    \caption{Hardware utilisation of SoC with 1 "big" core and 1 "small" core}
    \label{fig:b1s1-util}
\end{figure}

\subsubsection{Core Designs}
The "big" core implements the RV64IMAFDZicsrZifenciC\cite{riscv-1} instruction set extensions. The IMAFDZicsrZifenci are commonly shortened to G, and are instructions for integer multiplication and division, atomic instructions, single-precision floating-point arithmetic, double-precision floating-point arithmetic, control and status registers and instruction-fetch fence. The final extension, C, is for compressed instructions, allowing the width of an instruction to be just 16 bits instead of the standard 32 for certain instructions. 

The "small" core reduces this to RV64IAZicsrZifenciC\cite{riscv-1}, removing multiplication and both types of floating-point arithmetic. This reduces the capability of the core significantly, and means programs compiled for the "big" core that utilise the extensions unique to the "big" core will not run on the "small" core. This does not follow the original project idea exactly, as the cores were to be equal/very similar in ability to execute instructions, but dissimilar in total processing power due to other factors like pipelining, cache, superscalar, etc. However, this has been unavoidable due to the inability to source a new FPGA in the time frame.

Currently, these cores are the standard designs included in RocketChip\cite{rocketchip}. However, there are future plans to modify the cores to be better suited to the tasks to be done on the SoC and the size of the available FPGA.

\subsection{Software}
\subsubsection{Linux}
Debian Linux was successfully run on a design implementing a single "big" RV64 rocketcore, and was used for tasks like internet messaging using IRC\cite{irc} and basic text editing. The FPGA was connected to using Minicom\cite{minicom}, a USB serial CLI tool.

IMAGINE THERES A SCREENSHOT OF A LINUX TERMINAL HERE

The "big" RV64 rocketcore is the minimum required to run mainstream Linux, as smaller designs forgo the memory-management unit (MMU). When present, the MMU is responsible for the transfer for data between the registers and memory, as well as ensuring only valid/safe memory addresses are used. When an MMU is not present, there are no checks on the memory that a program is accessing. This is a massive security issue if allowed in the OS, especially in a mainstream OS that could be attacked by malware. As such, mainstream Linux is not compatible with a processor without an MMU\cite{linux-memory} and therefore not with any processor containing a core smaller than the RV64 "big" rocketcore. As the Nexys A7 FPGA supports a single "big" rocketcore, Debian Linux\cite{debianriscv} was successfully run on an SoC design containing a single "big" rocketcore. This demonstrates that a processor containing cores equal or larger than the "big" rocketcore would be able to run mainstream linux.

The MMU requirement blocks the objective of running the RISC-V Debian Linux distribution on a custom heterogenous SoC, as the larger Nexys Video FPGA has not yet been sourced and will not be for several months. As such, the aims for software to run on the produced SoC have been reduced to bare-metal, with the stretch objective of embedded linux in the future. Embedded linux would not require an MMU\cite{linux-memory}, but would provide an extremely small amount of features when compared to Debian Linux, and have stringent constraints on the software running.

\subsubsection{Bare-metal software}
Bare-metal code has successfully been written, compiled and run on various SoC designs implemented on the FPGA. A design containing a single "big" rocketcore was implemented on the FPGA, and an SD card inserted into the board containing an ELF executable that was then run by the SoC after boot. The code outputted text over a serial connection that was then read by the host computer, and identified the 'hart' (hardware thread) of the core.

\begin{figure}[h!]
    \centering
    \includegraphics[scale=1]{./img/bare-metal-hart-id.png}
    \caption{Serial output of bare-metal program}
    \label{fig:bare-metal-hart}
\end{figure}

This was followed by attempting to run on the heterogeneous design previously mentioned. The only serial output received only identified a hart of 0, indicating only a single core was running the ELF executable. Investigation of the default bootrom, assembly code that sets up the core and starts the user executable, revealed that only hart's with an id of 0 started running user code - else they were trapped in a loop of waiting for an inter-processor interrupt to indicate they should execute code in a region of memory stored in a local register. The solution for this is to generate an interrupt and load the register with the location of the ELF, or to adjust the bootrom so that harts always branch to user code. In-line assembly in the C code to generate the interrupt and load the register is currently being worked on.

The current bare-metal software has all been written in C, and then cross-compiled using GCC for RISC-V. Rust is also a candidate language for bare-metal RISC-V programming\cite{rust-riscv}, and comes with certain benefits that improve usability, such as good type checking and memory safety. Changing from using C to Rust is currently being considered, and will be tested.
% \section{Project management}
%Include a timetable (in 2 week chunks) for the remainder of the academic year, up until the submission deadline. What has been achieved, and was it expected?
The progress made so far has been mixed - some objectives have been achieved much faster than was expected, while other objectives have taken a significantly longer amount of time than expected. This has been tracked effectively using the Trello\footnote{https://trello.com/} board, and allowed for the project supervisor to be updated frequently as to how the project has been progressing. Project meetings have occurred most weeks and have been useful in informing the project supervisor of progress, as well as getting advice for any issues that arose during the week.

The initial design of the SoC took only a single week, far under the initial estimate. This was mostly due to underestimating the capabilities of the RocketChip generator. Verifying and testing the design is ongoing, as difficulty has been had in creating software that will run on the SoC after implementation on the FPGA. There has been a much smaller amount of time available for the project than planned for, and has lead to unsatisfactory progress. To rectify this, the scope has been adjusted to better fit what is possible in the remaining time, and to fit the constraints of what is known to be possible now more is known of the subject area. In addition, much more work will be complete over the December period. Several weeks were allocated to be free or have minimal work complete, but will now be used to achieve progress that should have been made previously.

Due to the issues with obtaining an FPGA that would allow mainstream Linux to be tested, the project objectives have changed so the embedded Linux is the target instead. The objective for running embedded Linux has also been moved to 'could' in figure \ref{fig:objectives}, indicating that there is a medium likelihood this objective will not be completed.

\newgeometry{left=1cm,bottom=1cm}
\begin{landscape}
    \begin{figure}[]
        \centering
        \includegraphics[scale=0.6]{./img/Gantt progress report.drawio.png}
        \caption{Gantt chart for progress tracking}
        \label{fig:Gantt}
    \end{figure}
\end{landscape}
\restoregeometry
The Gantt\cite{gantt-chart} chart in figure \ref{fig:Gantt} shows the progress that has been made so far, as well as expected tasks over the next few weeks. There is a significant amount of time between the presentation and the final report due date, and this has been left mostly empty. This is to accommodate any issues that arise during the project, and should allow time for more to be done between the presentation and report submission if necessary.

\bibliographystyle{../common/plainnat}
\bibliography{../common/bibliography}

\end{document}
